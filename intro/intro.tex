% This sample file is dedicated to the public domain.
\chapter{Introduction} \label{c.intro}
The subject of this dissertation concerns the development of agents and methodologies to enhance the sensivity of the magnetic resonance (MR) techniques, which includes both nuclear magnetic resonance (NMR) spectroscopy as well as magnetic resonance imaging (MRI). As will be shown later, MRI is a general extension of the principles of NMR. Hence, while the discussion of MR physics will largely center on NMR, it can be safely assumed that anything which applies to NMR also applies to MRI. Any exceptions will be explicitly noted in the text. After a annotated history of MR to lend context to the current work, the focus of this chapter will shift to a cursory overview of the underlying principles of NMR and the mechanics of an NMR experiment. 

In Chapter~\ref{c.xenon}, the noble gas xenon will be introduced, including it's physicochemical properties, prior use for NMR, and advantages and disadvantages over conventional NMR techniques. The concept of chemical exchange saturation transfer (CEST), central to the mechanism by which \xe\ contrast agents are detected, will also be introduced. This chapter will serve as motivation for the development of new agents for \xe\ NMR which appear in later chapters.

Chapter~\ref{c.model} describes the effort to create a viable model for understanding contrast produced by \xe\ NMR agents. Based on a well-known set of differential equations that capture the time-dependent behavior of magnetization as classical vectors, the goal of this project was to produce a model which could ultimately be used for optimization of current agents, or guide the search for better ones.

Chapters~\ref{c.m13, c.nanoemulsion, c.gv} each describe the development of a new \xe\ contrast agent based on unique platforms. Relevant background, characterization, and experimental details are covered.

Chapters~\ref{c.fd, c.gv_app} discuss applications of two of the aforementioned platforms including actively-targeted molecular imaging, and contrast under genetic control.

Finally, Chapter~\ref{c.conclusion} discusses future directions and applications.

\section{A brief history of nuclear magnetic resonance}

The phenomenom of nuclear magnetic resonance was discovered in 1945. Felix Bloch, of Harvard, and Ed Purcell, of Stanford, were awarded the Nobel Prize in Physics in 1953. In independent experiments, Bloch was able to show xxx, while Purcell found yyy. By and large, MR remained the province of physicists until 1960s.    

This is some stuff about science. I feel a strange compulsion to cite
\cite{williams12}.

\section{Origins of magnetic resonance}

\subsection{Nuclear spin}
\subsection{Magnetization}
\subsection{Detecting magnetization}
\

%\input{intro/processed.tex}

